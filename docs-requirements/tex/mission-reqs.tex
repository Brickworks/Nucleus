%!TEX root = ../docs-requirements.tex
%% *************************************************************************
%% TerraMICRO
%% Mission and System Requirements Specification
%%
%% This document shall list and describe all mission requirements, their 
%% criteria for success, and (when applicable) methods by which these 
%% requiremnts shall be evaluated.
%%
%% The intent of this specification is to quantify and control the criteria
%% by which mission success is defined, and to provide traceability to each
%% subsystem's performance to ensure mission success is achieved by the 
%% vehicle's design.
%%
%% Created November 26, 2018 by Phil Linden
%%
%% DEPENDENCIES
%%  \usepackage{tabularx}
%%  \usepackage{hyperref}
%% *************************************************************************
\section{Mission Requirements}
\subsection{Key Design Requirements}
Regardless of the mission objectives, the HAB system must meet several key 
design requirements in order to achieve mission success. These requirements
serve as success criteria and also as constraints to the design trade space.

The flight vehicle \textbf{must}:
\begin{itemize}
    \item Include multiple independent cut-down mechanisms.
    \item Include redundant tracking systems.
    \item Achieve 3 to 6 hours of powered flight time
    \item Have a total mass of no more than 2.72\,kg (6\,lbs.).
\end{itemize}

\subsection{Engineering \& Technology Objectives}
\label{mission-reqs:eng-objectives}
The following objectives provide the basis for TerraHAB's criteria for success
and drive all other mission requirements. These objectives steer the vision
and end goals for the mission and every subsystem or feature in the end result 
should support at least one of these objectives.

\begin{table}[h!]
    \centering
    \begin{tabularx}{\textwidth}{lX}
        \textit{In-Flight Balloon Monitoring} & A system will be able to monitor 
        the  temperature and pressure within a balloon and report that back to the 
        HAB payload. \\

        \textit{\textmu HAB Avionics Platform} & Flight test \textmu HAB as a 
        flexible, expandable, and cost-effective platform to support many 
        different mission profiles or payloads with all of the basics
        for a HAB launch included out of the box. \\ % \cite{dans microhab design doc} 

        \textit{Open-Loop Altitude Regulation} & Limit maximum altitude and rate of 
        ascent by the controlled release of helium during flight to prolong the 
        mission duration. Maintain 75,000 feet altitude for at least 30 minutes. \\

        \textit{HD On-Board Video} & Horizon-looking full color video at 
        1080p\@30 fps or better (1080p\@60fps or 4K\@30fps preferred). \\

        \textit{Video Capture of Balloon Burst} & Capture the balloon burst 
        event with minimum resolution of 720p\@60fps or better. 
        (720p@120fps or 1080p@120fps preferred) \\
    \end{tabularx}
\end{table}

\subsection{Stretch Goals \& Desired Features}
There are several design features that are specific requests from TerraHAB
engineers. The flight system should meet these requests or provide justification
for not including them. These features are not required for mission success
as defined in \autoref{mission-reqs:eng-objectives}, but it is expected that
the TerraHAB team strives to accomplish these goals.

\begin{table}
    \centering
    \begin{tabularx}{\textwidth}{lX}
        \textit{``Remove Before Flight'' Pins} & Include externally accessible remove before
        flight pins to safe or disarm subsystems while on the ground, such as a power 
        pin (included in \textmu HAB), startup sequence pin, launch pin, etc. \\
        \textit{Status Inticators \& Displays} & Include displays and self-test and status check codes to ensure that
        the balloon is stable and behaving nominally for flight. \\
        \textit{Simple Balloon Filling} & Simple and clear procedures during flight 
        preparations, including a quick-disconnect from helium fill plumbing. \\
        \textit{Vegetation Density Experiment} & Use NDVI with a commercially 
        available RGB (VNIR) camera to estimate vegetation density from images 
        in real time during flight. Minimum video quality 480p\@30fps. \\
    \end{tabularx}
\end{table}
